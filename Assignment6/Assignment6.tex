\documentclass{article}
\usepackage[utf8]{inputenc}
\usepackage{array}
\usepackage{multicol}
\usepackage{listings}
\usepackage{amssymb}
\usepackage{enumitem}
\usepackage{graphicx}
\usepackage{amsthm}
\usepackage{hyperref}
\usepackage{tikz}
\usepackage{amssymb}
\usepackage{amsmath}
\usepackage{mathtools}

\begin{document}
\title{Functional Programming \\ Exercise set 6}
\date{\today}
\author{Tony Lopar s1013792 \\ Carlo Jessurun s1013793 \\ Marnix Dessing s1014097}
\maketitle

\section*{Exercise 1}
Code can be found in BinaryTree.lhs\\
The instance declarations aren't very useful. Ordering a binary tree is not a very useful action... Most of the time we are interested in what is in the binary tree and not if it is bigger or smaller in comparison with another tree. Besides that the definition of bigger or smaller is not really clear, is it the number of nodes? The total sum of values?\\
The Eq instance declaration could be more useful, a use case could be: A tree is the result from some sort of user input, and checking if the tree is the same as the original tree given to the user, could tell if the input is changed from the original data, and some action is now required... 

\section*{Exercise 2}
\section*{Exercise 3}
\section*{Exercise 4}
\section*{Exercise 5}
\section*{Exercise 6}


\end{document}
