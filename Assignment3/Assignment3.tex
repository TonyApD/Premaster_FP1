\documentclass{article}
\usepackage[utf8]{inputenc}
\usepackage{array}
\usepackage{multicol}
\usepackage{listings}
\usepackage{amssymb}
\usepackage{enumitem}
\usepackage{graphicx}
\usepackage{amsthm}
\usepackage{hyperref}
\usepackage{tikz}
\usepackage{amssymb}
\usepackage{amsmath}
\usepackage{mathtools}

\begin{document}
\title{Functional Programming \\ Exercise set 3}
\date{\today}
\author{Tony Lopar s1013792 \\ Carlo Jessurun s1013793 \\ Marnix Dessing s1014097}
\maketitle

\section*{Exercise 1}
Thsi is implemented in $WordList.lhs$.\\
``Can you format the output so that one entry is shown per line?'' Sure.

\section*{Exercise 2}
This exercise can be found in the attached file $Pattern.lhs$.

\section*{Exercise 3}
The implementation of the run function may be found in the attached file $Run.lhs$.

The partitioned list already contains partially sorted lists. So, when we want to sort the list, we already know that some parts are already sorted which decreases computing time.

\section*{Exercise 4}


\section*{Exercise 5}
\begin{enumerate}
  \item The implementation can be found in the attached file \emph{Quicktest.lhs}. In order to check whether the next element in non-decreasing we should check that the next value is higher of equal to the previous.
  \item The function is defined as permutations. The number of factorations is equal to the factorial of n, so permutations n will result a list of length n!.
  \item A possible testing procedure could be to check for each generated run whether this is sorted in a non-decreasing order.
  \item We can use the fact that $\sqrt{n} \cdot \sqrt{n} = n$ to check whether the right square root has been computed.
  \item If the first list contains n and the second list m elements there will be n x m combinations possible, since every value from n can be combined with all m. So if we have bools = [True, False] and chars = ``Tony'' we will have 8 pairs in a list as result when we execute $bools \otimes chars$ since bools contains 2 and chars contains 4 elements. The output will be:  [(True,'T'),(True,'o'),(True,'n'),(True,'y'),(False,'T'),(False,'o'),(False,'n'),(False,'y')]
\end{enumerate}

\section*{Exercise 6}
The implementation of the run function may be found in the attached file $WordList.lhs$. I've done this in the same file after rule 41 to reuse the lorem text but eventually decided to make a new one anyway. Due to these optimalisations format can be called with ``format 40 loremConcat'' to see the output.


\end{document}
