\documentclass{article}
\usepackage[utf8]{inputenc}
\usepackage{array}
\usepackage{multicol}
\usepackage{listings}
\usepackage{amssymb}
\usepackage{enumitem}
\usepackage{graphicx}
\usepackage{amsthm}
\usepackage{hyperref}
\usepackage{tikz}
\usepackage{ulem}
\usepackage{amssymb}
\usepackage{amsmath}
\usepackage{mathtools}

\begin{document}
\title{Functional Programming \\ Exercise set 1}
\date{\today}
\author{Tony Lopar s1013792 \\ Carlo Jessurun s1013793 \\ Marnix Dessing s1014097}
\maketitle

\section*{Exercise 1}
The solutions for the exercises from 1 to 5 can be found in the Database.lhs file.

The following expressions are made for 6:
\begin{enumerate}[label=\alph*)]
  \item increment the age of all students by two: \\
  $map (\backslash p -> (age \enspace p+2, name \enspace p)) students$
  \item promote all of the students (attach ``dr '' to their name): \\
  $map (\backslash p -> (age \enspace p, name \enspace p \enspace ++ \enspace "dr")) students$
  \item find all students named Frits: \\
  $filter(\backslash p -> name \enspace p == "Frits") students$
  \item find all students whose favourite course is Functional Programming: \\
  $filter(\backslash p -> favouriteCourse \enspace p == "Functional \enspace Programming") students$
  \item find all students who are in their twenties: \\
  $filter(\backslash p -> age \enspace p >= 20 \enspace \&\& \enspace age \enspace p < 30 ) students$
  \item find all students whose favourite course is Functional Programming and who are in their twenties: \\ $filter(\backslash p -> favouriteCourse \enspace p == "Functional \enspace Programming" \enspace \&\& \enspace age \enspace p >= 20 \enspace \&\& \enspace age \enspace p < 30) students$
  \item find all students whose favourite course is Imperative Programming or who are in their twenties: \\ $filter(\backslash p -> favouriteCourse \enspace p == "Imperative \enspace Programming" \enspace || \enspace age \enspace p >= 20 \enspace \&\& \enspace age \enspace p < 30) students$
\end{enumerate}

\section*{Exercise 2}
\begin{enumerate}
  \item We need to evaluate the expresion: $insertionSort (7 : (9 : (2 : [ ]))$.
  Firstly $(7: (9: (2: [ ])))$ gets evaluated resulting in the list $[7,9,2]$. Then insert gets called with 7 and the rest of the list $[9,2]$. At this point seven is indeed lesser then 9 so the list gets returned. $[7,9]$. In insertionSort however, insertinoSort was called with xs as well. This is now called with $[9,2]$. Insert is then called with 9 and a list of $[2]$. Now the second guarded statement gets executed where 9 is indeed bigger then 2. insert will get one recursive call where b get's pushed into the list resulting in $[2,9]$. Going back up the same will be done with the remaining 7 which will result in \textbf{Result:} $[2,7,9]$.
  \item So first we will evaluate twice $(+1) 0$. This can be seen as $(+1) ((+1) 0)$ where $(+1) 0$ gets evaluated first and the result will be used as input for the second run of $(+1)$ the result is obviously 2.
  The second statement $twice twice (*2) 1$ can be seen as $(*2) ((*2) ((*2) ((*2) 1)))$ which results in 16.
  output of the list of statements:
  \begin{enumerate}
    \item ``Double pipe''
    \item ``Quadruple pipes''
    \item ``Sixteen pipes''
    \item ``Loads of pipes''
    \item ``Sixteen pipes''
    \item ``Not going to count this `amount` pipes''
    \item ``Same amount as above `amount` of pipes''
  \end{enumerate}
  The functions run in $2^n$ where n is the number of function calls.
\end{enumerate}

\section*{Exercise 3}
Now the definition of twice has changed to: $twice = \backslash f \rightarrow \backslash x \rightarrow f (f x)$.

\section*{Exercise 4}

\section*{Exercise 5}

\section*{Exercise 6}
The solutions for this exercise can be found in the attached file Shapes.lhs.

\end{document}
