\documentclass{article}
\usepackage[utf8]{inputenc}
\usepackage{array}
\usepackage{multicol}
\usepackage{listings}
\usepackage{amssymb}
\usepackage{enumitem}
\usepackage{graphicx}
\usepackage{amsthm}
\usepackage{hyperref}
\usepackage{tikz}
\usepackage{ulem}
\usepackage{amssymb}
\usepackage{amsmath}
\usepackage{mathtools}

\begin{document}
\title{Functional Programming \\ Exercise set 1}
\date{\today}
\author{Tony Lopar s1013792 \\ Carlo Jessurun s1013793 \\ Marnix Dessing s1014097}
\maketitle

\section*{Exercise 1}
The solutions for the exercises from 1 to 5 can be found in the Database.lhs file.

The following expressions are made for 6:
\begin{enumerate}[label=\alph*)]
  \item increment the age of all students by two: \\
  $map (\backslash p -> (age \enspace p+2, name \enspace p)) students$
  \item promote all of the students (attach "dr " to their name): \\
  $map (\backslash p -> (age \enspace p, name \enspace p \enspace ++ \enspace "dr")) students$
  \item find all students named Frits: \\
  $filter(\backslash p -> name \enspace p == "Frits") students$
  \item find all students whose favourite course is Functional Programming: \\
  $filter(\backslash p -> favouriteCourse \enspace p == "Functional \enspace Programming") students$
  \item find all students who are in their twenties: \\
  $filter(\backslash p -> age \enspace p >= 20 \enspace \&\& \enspace age \enspace p < 30 ) students$
  \item find all students whose favourite course is Functional Programming and who are in their twenties: \\ $filter(\backslash p -> favouriteCourse \enspace p == "Functional \enspace Programming" \enspace \&\& \enspace age \enspace p >= 20 \enspace \&\& \enspace age \enspace p < 30) students$
  \item find all students whose favourite course is Imperative Programming or who are in their twenties: \\ $filter(\backslash p -> favouriteCourse \enspace p == "Imperative \enspace Programming" \enspace || \enspace age \enspace p >= 20 \enspace \&\& \enspace age \enspace p < 30) students$
\end{enumerate}

\section*{Exercise 2}
\begin{enumerate}
  \item We need to evaluate the expresion: $insertionSort (7 : (9 : (2 : [ ]))$.
  \item The functions run in $2^n$ where n is the number of function calls. (Something like this was shown in the tutorial)
\end{enumerate}

\section*{Exercise 3}
\begin{enumerate}
    \item Now the definition of twice has changed to: $twice = \backslash f \rightarrow \backslash x \rightarrow f (f x)$.\\
        And evaluate:
        \begin{enumerate}
            \item $twice (+1) 0 = 2$ 
            \item $twice \quad twice (*2) 1 = 16$ 
        \end{enumerate}
    \item Is it worrying that we can apply a function to itself?
\end{enumerate}

\section*{Exercise 4}

\section*{Exercise 5}

\section*{Exercise 6}
The solutions for this exercise can be found in the attached file Shapes.lhs.

\end{document}
