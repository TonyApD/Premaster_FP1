\documentclass{article}
\usepackage[utf8]{inputenc}
\usepackage{array}
\usepackage{multicol}
\usepackage{listings}
\usepackage{amssymb}
\usepackage{enumitem}
\usepackage{graphicx}
\usepackage{amsthm}
\usepackage{hyperref}
\usepackage{tikz}
\usepackage{amssymb}
\usepackage{amsmath}
\usepackage{mathtools}

\begin{document}
\title{Functional Programming \\ Exercise set 7}
\date{\today}
\author{Tony Lopar s1013792 \\ Carlo Jessurun s1013793 \\ Marnix Dessing s1014097}
\maketitle

\section*{Exercise 1}
If `a' is smaller or eqaul to `b', then `a' should be placed before `b' in the list. Otherwise we know that `b' is smaller, so we should compare it with the tail of the list to find the spot where `a' should be and keep `b` before it. \\
\textbf{case []} \\
  insert \enspace a [] = [a] \\
\textbf{case (b:xs) \& a $\leq$ b} \\
  insert \enspace a (b:xs) = a:b:xs \\
\textbf{case (b:xs) \& a $>$ b} \\
  insert \enspace a (b:xs) = b:insert \enspace a \enspace xs \\

The function insertionsort should try to place all the elements of the list in the right position.

\section*{Exercise 2}
\begin{enumerate}
  \item The datatype tree elem was defined as follows: \\
  data Tree elem = Empty $|$ Node (Tree elem) elem (Tree elem) \\
  \newline
  The induction scheme for this datatype holds for the value of Empty which is the base case. The induction step covers the case that Tree elem = Node (Tree elem) elem (Tree elem)
  \item
  \begin{minipage}[t]{1\textwidth}
  The functions inner and outer are defined as follows:
  \newline
  inner :: Tree elem $->$ Integer \\
  inner Empty = 0 \\
  inner (Node l a r) = 1 + inner l + inner r \\
  \newline
  outer Empty = 1 \\
  outer (Node l a r) = 0 + outer l + outer r \\
  \newline
  \textbf{Induction hypothesis}: inner t + 1 = outer t \\
  \textbf{case t = Empty : inner Empty + 1} \\
  inner Empty + 1 = outer t \\
  = $\{$definition of inner$\}$ \\
  0 + 1 = 1 \\
  = $\{$definition of outer$\}$ \\
  outer Empty \\
  \newline
  \textbf{case t = Node l a r }\\
  inner (Node l a r) + 1 \\
  = $\{$definition of inner$\}$ \\
  1 + inner l + inner r + 1 \\
  = $\{$induction assumption$\}$ \\
  = outer l + inner r + 1 \\
  = $\{$induction assumption$\}$ \\
  = outer l + outer r \\
  \newline
  outer l + outer r \\
  = $\{$definition of outer$\}$ \\
  outer (Node l a r)
  \end{minipage}
  \item
  \textbf{Inductive hypothesis:} $2^{minHeight \enspace t} - 1 <= size \enspace t <= 2^{maxHeight \enspace t} - 1$ \\
  First, we take a look at the definition of minHeight and we will show the left side of the hypothesis. \\
  minHeight :: Tree elem $->$ Int \\
  minHeight Empty = 0 \\
  minHeight (Node l a r) = 1 + (minHeight l `min` minHeight r) \\
  \newline
  \textbf{case t = Empty:} \\
  $2^{minHeight \enspace Empty} - 1$ \\
  = $\{$definition of minHeight$\}$ \\
  $2^0 - 1$ \\
  = $\{$arithmetic$\}$ \\
  0 \\
  = $\{$definition of size$\}$ \\
  size Empty \\
  \newline
  \textbf{case t = Node l a r:} \\
  $2^{minHeight \enspace (Node \enspace l \enspace a \enspace r)} - 1$ \\
  = $\{$definition of minHeight$\}$ \\
  $2^{1 + (minHeight \enspace l \enspace `min` \enspace minHeight \enspace r)} - 1$ \\
  $2 \cdot 2^{minHeight \enspace l \enspace `min` \enspace minHeight \enspace r}$ \\
  $2^{minHeight \enspace l \enspace `min` \enspace minHeight \enspace r} + 2^{minHeight \enspace l \enspace `min` \enspace minHeight \enspace r}$ \\
  $1 + 2^{minHeight \enspace l} - 1 + 2^{minHeight \enspace r} - 1$ \\
  $<=$ $\{$induction assumption$\}$ \\
  $1 + size \enspace l + size \enspace r$ \\
  = $\{$definition of size$\}$ \\
  size(Node l a r)

  Now, we continue with the function maxHeight. This function is defined as follows:\\
  maxHeight :: Tree elem $->$ Int \\
  maxHeight Empty = 0 \\
  maxHeight (Node l a r) = 1 + (maxHeight l `max` maxHeight r) \\
  \newline
  \textbf{case t = Empty:} \\
  $2^{maxHeight \enspace Empty} - 1$ \\
  = $\{$definition of maxHeight$\}$ \\
  $2^0 - 1$ \\
  = $\{$arithmetic$\}$ \\
  0 \\
  = $\{$definition of size$\}$ \\
  size Empty \\
  \newline
  \textbf{case t = Node l a r:} \\
  $2^{maxHeight \enspace (Node \enspace l \enspace a \enspace r)} - 1$ \\
  = $\{$definition of minHeight$\}$ \\
  $2^{1 + (maxHeight \enspace l \enspace `max` \enspace maxHeight \enspace r)} - 1$ \\
  $2 \cdot 2^{maxHeight \enspace l \enspace `max` \enspace maxHeight \enspace r}$ \\
  $2^{maxHeight \enspace l \enspace `max` \enspace maxHeight \enspace r} + 2^{maxHeight \enspace l \enspace `max` \enspace maxHeight \enspace r}$ \\
  $1 + 2^{maxHeight \enspace l} - 1 + 2^{maxHeight \enspace r} - 1$ \\
  $>=$ $\{$induction assumption$\}$ \\
  $1 + size \enspace l + size \enspace r$ \\
  = $\{$definition of size$\}$ \\
  size(Node l a r)
\end{enumerate}

\section*{Exercise 3}
\section*{Exercise 4}
% not(a && b) = nod a || not b
% not a . and = or map not
% not(and[a, b]) = or(map not [a, b]) = or [not a, not b]
% not(a && b) = not a || not b
%
% not(foldr(&&) True bs)
% = {fusion}
%   foldr(=>) (not True) bs
%
% foldr(||) False(map not bs)
% = {foldr-map fusion}
%   foldr(\ab -> not a || b) False bs
%
% not(a && b) = a =>(implies) not b
%
% and = foldr (&&) True
% or = foldr (||) False
\section*{Exercise 5}
\section*{Exercise 6}


\end{document}
