\documentclass{article}
\usepackage[utf8]{inputenc}
\usepackage{array}
\usepackage{multicol}
\usepackage{listings}
\usepackage{amssymb}
\usepackage{enumitem}
\usepackage{graphicx}
\usepackage{amsthm}
\usepackage{hyperref}
\usepackage{tikz}
\usepackage{amssymb}
\usepackage{amsmath}
\usepackage{mathtools}

\begin{document}
\title{Functional Programming \\ Exercise set 7}
\date{\today}
\author{Tony Lopar s1013792 \\ Carlo Jessurun s1013793 \\ Marnix Dessing s1014097}
\maketitle

\section*{Exercise 1}
If `a' is smaller or eqaul to `b', then `a' should be placed before `b' in the list. Otherwise we know that element `b' is smaller, so we should compare `a' with the tail of the list to find the spot where it should be and keep `b` before it. \\
\textbf{case [] (base case)} \\
  insert \enspace a [] = [a] \\
  When the list is empty, then adding an element will result in a sorted list. \\
\textbf{case (b:xs) \& a $\leq$ b (Inductive step scenario 1)} \\
  insert \enspace a (b:xs) = a:b:xs \\
  In this case the value of a should be put in front of the b in the list.\\
\textbf{case (b:xs) \& a $>$ b (Inductive step scenario 2)} \\
  insert \enspace a (b:xs) = b:insert \enspace a \enspace xs \\
  In this case the b should be before the value of a and we need to find the position where to put the value of a in the tail of the list.\\
  \newline
The function insertionsort returns an ordered list. The function has two cases.
\textbf{case [] (base case)} \\
  insertionsort \enspace [] = [] \\
  Since an empty list doesn't contain any element it's already sorted, so we should return the empty list.\\
\textbf{case (b:xs) \& a $\leq$ b (Inductive step scenario 1)} \\
  insertionSort \enspace (x : xs) = insert \enspace  x \enspace (insertionSort xs) \\
  In this case we try to insert the first element of the list in the already sorted tail xs. The tail will recurse until it's the empty list and then insert all elements in this list. Since the function insert makes sure that the element is inserted in a position in which the list is sorted, the call to insert will preserve the property that the list is sorted after every insert.

\section*{Exercise 2}
\begin{enumerate}
  \item The datatype tree elem was defined as follows: \\
  data Tree elem = Empty $|$ Node (Tree elem) elem (Tree elem) \\
  \newline
  The induction scheme for this datatype holds for the value of Empty which is the base case. The induction step covers the case that Tree elem = Node (Tree elem) elem (Tree elem)
  \item
  \begin{minipage}[t]{1\textwidth}
  The functions inner and outer are defined as follows:
  \newline
  inner :: Tree elem $->$ Integer \\
  inner Empty = 0 \\
  inner (Node l a r) = 1 + inner l + inner r \\
  \newline
  outer Empty = 1 \\
  outer (Node l a r) = 0 + outer l + outer r \\
  \newline
  \textbf{Induction hypothesis}: inner t + 1 = outer t \\
  \textbf{case t = Empty : inner Empty + 1} \\
  inner Empty + 1 = outer t \\
  = $\{$definition of inner$\}$ \\
  0 + 1 = 1 \\
  = $\{$definition of outer$\}$ \\
  outer Empty \\
  \newline
  \textbf{case t = Node l a r }\\
  inner (Node l a r) + 1 \\
  = $\{$definition of inner$\}$ \\
  1 + inner l + inner r + 1 \\
  = $\{$induction assumption$\}$ \\
  = outer l + inner r + 1 \\
  = $\{$induction assumption$\}$ \\
  = outer l + outer r \\
  \newline
  outer l + outer r \\
  = $\{$definition of outer$\}$ \\
  outer (Node l a r)
  \end{minipage}
  \item
  \textbf{Inductive hypothesis:} $2^{minHeight \enspace t} - 1 <= size \enspace t <= 2^{maxHeight \enspace t} - 1$ \\
  First, we take a look at the definition of minHeight and we will show the left side of the hypothesis. \\
  minHeight :: Tree elem $->$ Int \\
  minHeight Empty = 0 \\
  minHeight (Node l a r) = 1 + (minHeight l `min` minHeight r) \\
  \newline
  \textbf{case t = Empty:} \\
  $2^{minHeight \enspace Empty} - 1$ \\
  = $\{$definition of minHeight$\}$ \\
  $2^0 - 1$ \\
  = $\{$arithmetic$\}$ \\
  0 \\
  = $\{$definition of size$\}$ \\
  size Empty \\
  \newline
  \textbf{case t = Node l a r:} \\
  $2^{minHeight \enspace (Node \enspace l \enspace a \enspace r)} - 1$ \\
  = $\{$definition of minHeight$\}$ \\
  $2^{1 + (minHeight \enspace l \enspace `min` \enspace minHeight \enspace r)} - 1$ \\
  $2 \cdot 2^{minHeight \enspace l \enspace `min` \enspace minHeight \enspace r}$ \\
  $2^{minHeight \enspace l \enspace `min` \enspace minHeight \enspace r} + 2^{minHeight \enspace l \enspace `min` \enspace minHeight \enspace r}$ \\
  $1 + 2^{minHeight \enspace l} - 1 + 2^{minHeight \enspace r} - 1$ \\
  $<=$ $\{$induction assumption$\}$ \\
  $1 + size \enspace l + size \enspace r$ \\
  = $\{$definition of size$\}$ \\
  size(Node l a r)

  Now, we continue with the function maxHeight. This function is defined as follows:\\
  maxHeight :: Tree elem $->$ Int \\
  maxHeight Empty = 0 \\
  maxHeight (Node l a r) = 1 + (maxHeight l `max` maxHeight r) \\
  \newline
  \textbf{case t = Empty:} \\
  $2^{maxHeight \enspace Empty} - 1$ \\
  = $\{$definition of maxHeight$\}$ \\
  $2^0 - 1$ \\
  = $\{$arithmetic$\}$ \\
  0 \\
  = $\{$definition of size$\}$ \\
  size Empty \\
  \newline
  \textbf{case t = Node l a r:} \\
  $2^{maxHeight \enspace (Node \enspace l \enspace a \enspace r)} - 1$ \\
  = $\{$definition of minHeight$\}$ \\
  $2^{1 + (maxHeight \enspace l \enspace `max` \enspace maxHeight \enspace r)} - 1$ \\
  $2 \cdot 2^{maxHeight \enspace l \enspace `max` \enspace maxHeight \enspace r}$ \\
  $2^{maxHeight \enspace l \enspace `max` \enspace maxHeight \enspace r} + 2^{maxHeight \enspace l \enspace `max` \enspace maxHeight \enspace r}$ \\
  $1 + 2^{maxHeight \enspace l} - 1 + 2^{maxHeight \enspace r} - 1$ \\
  $>=$ $\{$induction assumption$\}$ \\
  $1 + size \enspace l + size \enspace r$ \\
  = $\{$definition of size$\}$ \\
  size(Node l a r)
\end{enumerate}


\section*{Exercise 3}
The implementations can be found in Insort.lhs\\
\begin{enumerate}
    \item Code implementation.
    \item The new implementation isn't more efficient in terms of running time, when functions are used to generate test trees. And run both the insort and insortCat with the same trees.
    \item We want to prove that the inorderCat (defivation) equals the inorder of a tree. We define the inorder as [tl, a , tr], where tl is the inorder of the left subtree and tr is the inorder of the right subtree. We take a as the inorder value from the tree node.
        \begin{verbatim}
        BASE CASE:
            => {inorderCat empty tree with empty list}
        [] = inorderCat Empty []
            => {definition: inorderCat Empty xs = xs}
        []
            => {gives back just the empty list}
        
        INDUCTIVE STAP:
            => {inorder a tree with subtrees tl and tl }
        [tl,a,tr] = inorderCat (Node tl a tr) []
            => {definition inorderCat}
        [tl,a,tr] = (inorderCat tl (a:inorderCat tr []))
            => {write inorderCat tr as tr}
        [tl,a,tr] = (inorderCat tl (a:tr))
            => {prepend a to the inorderCat tr}
        [tl,a,tr] = (inorderCat tl [a,tr])
            => {write inorderCat tl as tl}
        [tl,a,tr] = (tl [a,tr])
            => {use definition of inorderCat}
        [tl,a,tr] = [tl,a,tr]
        \end{verbatim}
    
    \item Code implementation. 
    \item I do not know that Huges list is. From the slides §6.1 and §6.2 it does not become clear ether. And Google does not know any Huges.
\end{enumerate}


\section*{Exercise 4}
% not(a && b) = nod a || not b
% not a . and = or map not
% not(and[a, b]) = or(map not [a, b]) = or [not a, not b]
% not(a && b) = not a || not b
%
% not(foldr(&&) True bs)
% = {fusion}
%   foldr(=>) (not True) bs
%
% foldr(||) False(map not bs)
% = {foldr-map fusion}
%   foldr(\ab -> not a || b) False bs
%
% not(a && b) = a =>(implies) not b
%
% and = foldr (&&) True
% or = foldr (||) False

\section*{Exercise 5}
\begin{enumerate}
  \item The implementation is quite slow since it adds i-times an indent. In this indent there is also a recursion to tab.  After these indents x is concatinated to this.
  \item
\end{enumerate}

\section*{Exercise 6}


\end{document}
