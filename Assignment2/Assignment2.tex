\documentclass{article}
\usepackage[utf8]{inputenc}
\usepackage{array}
\usepackage{multicol}
\usepackage{listings}
\usepackage{amssymb}
\usepackage{enumitem}
\usepackage{graphicx}
\usepackage{amsthm}
\usepackage{hyperref}
\usepackage{tikz}
\usepackage{ulem}
\usepackage{amssymb}
\usepackage{amsmath}
\usepackage{mathtools}

\begin{document}
\title{Functional Programming \\ Exercise set 2}
\date{\today}
\author{Tony Lopar s1013792 \\ Carlo Jessurun s1013793 \\ Marnix Dessing s1014097}
\maketitle

\section*{Exercise 1}
\begin{enumerate}
  \item There are four functions possible, since the a Bool has two possible values. For both possible inputs there is a function that gives a for both one value. The functions that may exist are a function that only gives true, a function that only gives true, the identity function and the inverse. The defined functions may be found in the attached file Exercise1.lhs.
  \item There are 8 functions possible. Since there are 4 possible combinations possible as input. These four combinations all have two possible outputs. The defined functions may be found in the attached file Exercise1.lhs.
  \item The type $Bool \rightarrow Bool \rightarrow Bool$ may be a function that takes two parameters as input and returns a boolean or it may be a function that takes one Bool as input and returns a function from Bool to Bool. In the case of two values as input we have eight options as seen in the previous question. In the case that we return a funciton from Bool to Bool we also have 8 possibilities since there are two possible inputs and four function possible from Bool to Bool. This means that $Bool \rightarrow Bool \rightarrow Bool$ may have 16 different functions in total.
\end{enumerate}

\section*{Exercise 2}
The functions from this exercise are defined in the attached file Char.lhs. In this exercise we also had to decode the following message:

"MHILY LZA ZBHL XBPZXBL MVYABUHL HWWPBZ JSHBKPBZ JHLJBZ KPJABT HYJUBT LZA ULBAYVU"

With our implemented shift function we were able to recover the following message:

"FABER EST SUAE QUISQUE FORTUNAE APPIUS CLAUDIUS CAECUS DICTUM ARCNUM EST NEUTRON"

This message is found by shifting all the characters 19 characters. This is done by the following command:
$map (\backslash x \enspace -> \enspace shift \enspace 19 \enspace x)msg$.

The function will give the same result when the 19 is replaced by an integer x for which holds that $19 \equiv x `\textbf{mod}` 26$.

\newpage
\section*{Exercise 3}
Repeat the exercise using the type annotation ::Integer. What do you observe? Can you explain the differences?\\
\newline
The interesting output:
\begin{lstlisting}
Prelude: product [1..66] :: Int
0
Prelude: product [1..66] :: Integer
544344939077443064003729240247842752644293064388798874532860126869671081148416000000000000000
\end{lstlisting}
The Int type is defined as ``A fixed-precision integer type with at least the range $[-2^{29} \dots. 2^{29-1}]$.'' $[0]$ The range varies by machine but you can find it with minBound and maxBound
\newline
The reason it overflows is that it's got a fixed amount of memory allocated for it. Imagine a simpler example - a positive integer in memory whose maximum value was 7 (stored as 3 bits in memory)
\newline
\newline
0 is represented as 000, in binary\\
1 is represented as 001\\
2 is represented as 010, etc.\\
\newline
Note how the bit math works: when you add 1, you either turn the smallest digit from 0 to 1, or you turn it from 1 to 0 and do the same operation on the next-most-significant digit.
\newline
\newline
Now if you naievely (as Haskell does) perform this when there's no next-most-significant digit, then it just increments as usual, but ``gets its head cut off.'' E.g. 111 + 1 becomes 000 instead of 1000. This is what's happening with Haskell, except that its lowest value (represented as a series of 0s) is its lowest negative number. it uses its ``leftmost'' bit to represent +/-. You'll need to do (maxBound :: Int) + (maxBound :: Int) + 2 to get 0.
\begin{lstlisting}
>  maxBound :: Int
9223372036854775807
>  (maxBound :: Int) + 1
-9223372036854775808
>  (maxBound :: Int) + 2
-9223372036854775807
>  let x = (maxBound :: Int) + 1 in x + x
0
\end{lstlisting}
Why ``allow'' this to happen? Simple - efficiency. It's much faster to not check for if there'll be an integer overflow. This is the reason that Integer exists - it's unbounded, for when you think you might overflow, but you pay a price in efficiency.


\section*{Exercise 4}
Implemented in Exercise4.lhs.
The first two assignments where swap :: (Int, Int) (Int, Int) and swap :: (a, b)  (b, a) are implemented all work perfectly.
\newline
\newline
The difference is that the first pair ``(Int, (Char, Bool))'' consists of an int followed by a tuple of char and bool and the second pair ``(Int, Char, Bool)'' consists of the types int, char bool separately.
As for  the last question about datatypes, we can use a plethora of functions to achieve this. For example: \\
\textbf{fromIntegral} - to convert between say Int and Double\\
\textbf{pack / unpack} - to convert between ByteString and String\\
\textbf{read} - to convert from String to Int
\newline
An example implementation from Hoogle:
\begin{lstlisting}
fromIntegral = fromInteger . toInteger

{-# RULES
"fromIntegral/Int->Int" fromIntegral = id :: Int -> Int
  #-}

{-# RULES
"fromIntegral/Int->Word"  fromIntegral = \(I# x#) -> W# (int2Word# x#)
"fromIntegral/Word->Int"  fromIntegral = \(W# x#) -> I# (word2Int# x#)
"fromIntegral/Word->Word" fromIntegral = id :: Word -> Word
  #-}

-- | general coercion to fractional types
realToFrac :: (Real a, Fractional b) => a -> b
{-# NOINLINE [1] realToFrac #-}
realToFrac = fromRational . toRational
\end{lstlisting}


\section*{Exercise 5}
\begin{enumerate}
    \item Which of the following expressions are well-typed and well-formed?
    \begin{enumerate}
        \item $(+4)$, is typed: Num a $=> a \rightarrow a$ so it is well-typed, but not well-formed because the statement +4 is not applied to something but rather decibels a function. $(+4) 4$ would be well-formed because this results in 8. If we feed $(+4)$ to the compiler it says: No instance for (Show (Integer $\rightarrow$ Integer))
        \item $div$, is typed: $Integer \rightarrow Integer \rightarrow$. And so the same as above hold, it is well-typed, but it is not a valid expression i.e. not well-formed because there are missing two parameters.
        \item $div 7$, is the same as $div$ above, but now there is only missing one argument. Still well-typed, and not well-formed.
        \item $(div 7) 4$ div now has the two arguments and is a well-typed and well-formed expression.
        \item $div (7 \quad 4)$ div now has the two arguments and is a well-typed and well-formed expression.
        \item $7 `div` 4$ is infix notation for the above $(div 7) 4$.
        \item $+ 3 \quad 7$ is NOT well-formed, the $+$ operator should be used infix.
        \item $(+) 3 \quad 7$ is well-formed and well-typed, the $+$ operator is now in prefix notation, and forms a function with signature $(+)::$Num $a => a \rightarrow a \rightarrow a$
        \item b has the type Bool, $(b, `b`, "b")$ is not well-typed or formed.
        \item $(abs, `abs`, "abs")$ is not well-typed or formed.
        \item $abs \circ negate$, is typed: Num $=> c \rightarrow c$ and is well-typed but not well-formed the function needs an parameter.
        \item $(*3) \circ (+3)$ what holds for $abs \circ negate$ holds for this expression too. The functions abs and negate are replaced by *3 and +3 respectively.
    \end{enumerate}
    \item The other two functions:
    \begin{enumerate}
        \item $(abs \circ) \circ (\circ negate)$
        \item $(div \circ) \circ (\circ mod)$
    \end{enumerate}
    \item Infer the types of the following functions:
    \begin{enumerate}
        \item $i \quad x = x$, has the function signature $i::x \rightarrow x$, any type is accepted and the same type is returned. Also called the identity function.
        \item $k (x, y) = x$ has the signature $k::(x , y) \rightarrow x$ any tuple with any two types is accepted and the first tuple element is returned. So that is the type x.
        \item $b (x, y, z) = (x \quad z) y$ has the signature: $b :: (z \rightarrow y \rightarrow x, y, z) \rightarrow x$, this means x is a function from y and z to the return type of x.
        \item $c (x, y, z) = x (y z)$ has the signature: $c::(x \rightarrow y, z \rightarrow x, z) \rightarrow y$, this means y is applied to z, then x is applied to the result of y, the return type of x is the return type of the whole function.
        \item $s (x, y, z) = (x z) (y z)$ has the signature: $s::(t1 \rightarrow t2 \rightarrow t3, t1 \rightarrow t2, t1) \rightarrow t3$
    \end{enumerate}
\end{enumerate}

\section*{Exercise 6}
The functions that are made for this exercise can be found in the attached file Question6.lhs.
\begin{enumerate}
  \item We may define an infinite amount of functions from Int to Int, since the range is infinite. Since the type a can be any type the only total function is a the identity function.
  \item Many of the cases here were similar to predicate logic. This means we worked on similar exercise before.
  \item There are again infinite possible functions which take two integers and result an int. For example the function of addition takes two values and returns one result. The only possible function $(a -> a) ->a$ is the identity function since a can be of any type.
\end{enumerate}

\end{document}
