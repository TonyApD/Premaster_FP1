\documentclass{article}
\usepackage[utf8]{inputenc}
\usepackage{array}
\usepackage{multicol}
\usepackage{listings}
\usepackage{amssymb}
\usepackage{enumitem}
\usepackage{graphicx}
\usepackage{amsthm}
\usepackage{hyperref}
\usepackage{tikz}
\usepackage{ulem}
\usepackage{amssymb}
\usepackage{amsmath}
\usepackage{mathtools}

\begin{document}
\title{Functional Programming \\ Exercise set 2}
\date{\today}
\author{Tony Lopar s1013792 \\ Carlo Jessurun s1013793 \\ Marnix Dessing s1014097}
\maketitle

\section*{Exercise 1}
\begin{enumerate}
  \item There are four functions possible, since the a Bool has two possible values. For both possible inputs there is a function that gives a for both one value. The functions that may exist are a function that only gives true, a function that only gives true, the identity function and the inverse. The defined functions may be found in the attached file Exercise1.lhs.
  \item There are 8 functions possible. Since there are 4 possible combinations possible as input. These four combinations all have two possible outputs. The defined functions may be found in the attached file Exercise1.lhs.
\end{enumerate}

\section*{Exercise 2}


\section*{Exercise 3}


\section*{Exercise 4}


\section*{Exercise 5}


\section*{Exercise 6}


\end{document}
